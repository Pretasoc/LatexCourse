\chapter{Vorwort}
    Nachdem ich nun mitlerweile mehrere Jahre mit \LaTeX{} gearbeitet habe und ich immer wieder in Situationen gekommen bin, in dehnen ich anderen die Vorzüge von \LaTeX{} zwar aus meiner Sicht wunderbart darlegen konnte, aber wenn ich das Interesse geweckt hatte, immer an den Punkt kam, dass ich als Einstieg nur empfehlen konnte: \glqq \textit{Such dir einfach ein Tutorial. Wenn du dannach googelst findest du schon etwas passendes.}\grqq Das war sowohl für mich, als auch meist für den Gegenüber unbefriedigend, da es viele Gute einstiege gibt, mit dehnen ich aber verschiedene Probleme haben. Einige Vereinfachen Vorgehensweisen meiner Meinung nach zu sehr, dass der Leser eher schlechteren Code schreibt, ohne dass diese Vereinfachungen später korrigiert werden. Andere Geben einen guten Überblick lassen den Lernenden aber viel zu schnell alleine, so dass er wieder neu nach Hilfe suchen muss. Andere sind einfach nicht mehr auf dem aktuellen Stand. Zuletzt wäre da noch das Problem überhaupt ein Tutorial zu finden. Es gibt Tutorials nur für bestimmte Betriebssysteme. Es gibt Tutorials die eine bestimmte \LaTeX{}-Distribution vorraussetzen. Und ich habe fast kein Tutorial gefunden, dass Personen, die noch nie mit der einer Konsole bzw. einem Terminal gearbeitet haben, hilft mit dieser Art von Programm umzugehen. Ein ähnliches fehlen habe ich bei der Diskussion verschieder Editoren festgestellt.
    
        Dies sind zuerst einmal hohe Ansprüche an mein Eigenes Tutorial, aber ich werde mein Bestes geben sie zu erfüllen. Da ich in diesem Buch viele der \glqq üblichen\grqq{} Problemstellungen in \LaTeX{} lösen werde, habe ich mich entschlossen den Quelltext dieses Buches auf \href{https://github.com/pretasoc/latexcourse}{GitHub\footnote{\href{https://github.com/pretasoc/latexcourse}{https://github.com/pretasoc/latexcourse}}} zu veröffentlichen. Dies soll jedem die möglichkeit geben, nachzusehen, wie ich bestimmte Probleme gelöst habe; verbesserungen einzubringen und dabei mitzuhelfen diesen Einstieg aktuell zu halten.

            Was ich während meiner Bisherigen Arbeit mit \LaTeX{} gelernt habe ist, dass es in bezug auf \LaTeX{} öfter verschiedene Meinungen gibt, was der \glqq{}richtige\grqq{} Weg ist ein Problem zu lösen. Sofern ich von mehreren Wegen weiß, werde ich alle Beschreiben und erklären, welchen ich aus welchen Gründen bevorzuge. In einigen Fällen werde ich auch von bestimmten Lösungen abraten, sofern dies begründet ist.

            Philemon Eichin